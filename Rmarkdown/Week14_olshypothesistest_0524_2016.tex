\documentclass[xcolor=dvipsnames]{beamer}
\usepackage{amsfonts, epsfig, xspace, relsize}
\usepackage{algorithm,algorithmic, graphicx}
\usepackage{pstricks,pst-node}
\usepackage{multimedia}
\usepackage{enumerate}
\usepackage[normal,tight,center]{subfigure}
\setlength{\subfigcapskip}{-.5em}
\usepackage{beamerthemesplit}
\usepackage{mathtools}
\usepackage{xcolor}
\usepackage{fancyvrb}
\usepackage{framed,color}
\definecolor{shadecolor}{rgb}{0.8,1,0.5}
\usepackage{amsmath}% http://ctan.org/pkg/amsmath
\DeclareMathOperator*{\plim}{plim}
\usepackage[retainorgcmds]{IEEEtrantools}% http://ctan.org/pkg/ieeetran
\newcommand{\non}{\IEEEnonumber*}
\usepackage{scalerel}
\usepackage{pgfplots}
\usepackage{pgfplotstable}
\usepackage{filecontents}
\newcommand\reallywidehat[1]{%
\begin{array}{c}
\stretchto{
  \scaleto{
    \scalerel*[\widthof{#1}]{\bigwedge}
    {\rule[-\textheight/2]{1ex}{\textheight}} %WIDTH-LIMITED BIG WEDGE
  }{1.25\textheight} % THIS STRETCHES THE WEDGE A LITTLE EXTRA WIDE
}{0.5ex}\\           % THIS SQUEEZES THE WEDGE TO 0.5ex HEIGHT
#1\\                 % THIS STACKS THE WEDGE ATOP THE ARGUMENT
\rule{-1ex}{0ex}
\end{array}
}
\setbeamertemplate{caption}[numbered]
\usepackage{fontspec}  %加這個就可以設定字體
\usepackage{xeCJK}       %讓中英文字體分開設置
%\setromanfont{LiHei Pro} % 儷黑Pro
\newCJKfontfamily{\K}{標楷體}
\newCJKfontfamily{\H}{微軟正黑體}
\setmonofont[Scale=0.8]{Courier New} % 等寬字型
\setsansfont{Times New Roman}
\setCJKmainfont{YouYuan} %設定中文為系統上的字型,而英文不去更動,使用原TeX字型
\XeTeXlinebreaklocale "zh"             %這兩行一定要加,中文才能自動換行
\XeTeXlinebreakskip = 0pt plus 1pt     %這兩行一定要加,中文才能自動換行
%\usetheme{lankton-keynote}
\usetheme{Madrid}
\usecolortheme[named=BrickRed]{structure}
\author[蔡佳泓]{\K 蔡佳泓}

\title[Statistical Methods for Social Sciences]{Regression Analysis\\
\smallskip
{\small {Simple Linear Regression: Hypothesis Testing}}}
\date{May 24, 2016} %leave out for today's date to be insterted
\institute[ESC \& GIEAS]{\H 國立政治大學選舉研究中心與東亞研究所}
\begin{document}
\maketitle
\tableofcontents
\section{檢定的方式}
\begin{frame}{\H 迴歸係數的推論}
迴歸係數有三種推論方式:
\begin{enumerate}
\item 點估計:用$ \hat{\beta}_{1} $推估$ \beta_{1} $
\item 假設檢定:考慮抽樣分佈,用檢定統計檢定$ \beta_{1} $在$\alpha$信賴水準的大小
\item 區間估計:建構可以涵蓋$ \beta_{1} $的100(1-$\alpha$)\%的信賴區間。
\end{enumerate}
高斯馬可夫假設6已經告訴我們誤差項需要成常態分佈。但是樣本規模夠大時並不需要這個假設。
\end{frame}
\begin{frame}{\H 迴歸係數}
\begin{itemize}
\item 根據以上的迴歸假設,OLS迴歸係數有$\textcolor{blue}{\mathsf{BLUE}}$的特性。
\item 而在$\textcolor{blue}{\mathsf{CLM}}$假設成立的前提下,$ \beta_{1} $的抽樣分佈表示為:
\begin{center}
$ \hat{\beta}_{1} \sim N(\beta_{1},Var[\hat{\beta}_{1}|X])$\\
$\mathlarger{Var[\hat{\beta}_{1}|X]=\frac{\sigma_{u}^{2}}{\sum_{i=1}^{n}(x_{i}-\bar{x})^2} } $
\end{center}
$\beta_{0}$的抽樣分佈表示為:
\begin{center}
$\hat{\beta}_{0} \sim N(\beta_{0}, \mathlarger{\frac{\sigma_{u}^2\sum_{i=1}^{n}x^2}{n\sum_{i=1}^{n}(x-\bar{x})^2}} ) $
\end{center}
又可以寫成:
\begin{center}
$\hat{\beta}_{j}-\beta_{j}/\reallywidehat{SE}(\hat{\beta}_{j})\sim N(0,1)$
\end{center}
\end{itemize}
\end{frame}
\begin{frame}{\H 母體參數的$t$檢定}
\begin{itemize}
\item 因為$\hat{\beta}_{1}$的標準誤可表示為:\\
$SE[\hat{\beta}_{1}]=\mathlarger{\frac{\sigma_{u}}{\sqrt{\sum_{i=1}^{n}(x-\bar{x})^2}}}$\\
\item 其中 $\sigma_{u}$可用$\hat{\sigma}_{u}^2$估計,而且$\hat{\sigma}_{u}^2=\mathlarger{\frac{\sum_{i=1}^{n}\hat{u}_{i}^{2}}{n-2}}$
\item 所以$SE[\hat{\beta}_{1}]=\mathlarger{\frac{\frac{\sum_{i=1}^{n}\hat{u}_{i}^{2}}{n-2}}{\sqrt{\sum(x_{i}-\bar{x})^2}}}  $
\end{itemize}
\end{frame}
\begin{frame}{\H $t$值}
\begin{theorem}[t值的抽樣分佈]
$\beta_{1}$的$t$值呈現$t$分佈,自由度為n-2
\begin{center}
$T\equiv \hat{\beta}_{1}-\beta_{1}/\reallywidehat{SE}[\hat{\beta}_{1}]\sim \tau_{n-2}$
\end{center}
\end{theorem}
因為$t$分佈的右邊尾巴比常態分佈長,更適合用來檢定有許多不確定性的$T$值\\
\medskip
而且計算$\hat{\sigma}_{u}^2$使用($\hat{\beta}_{0},\hat{\beta}_{1}$),所以自由度是n-2
\end{frame}
\subsection{t檢定}
\begin{frame}{\H $t$檢定步驟}
已知:
\begin{center}
$T\equiv \hat{\beta}_{1}-c/\reallywidehat{SE}[\hat{\beta}_{1}]\sim \tau_{n-2}$
\end{center}
\begin{enumerate}
\item $c$是虛無假設的值,可以是0。
\item $T$值會有落在某個區域的機率$P$
\begin{center}
$P\big(-t_{\alpha/2}(n-2)\leq T \leq t_{\alpha/2}(n-2)\big)=1-\alpha$
\end{center}
其中$t_{\alpha/2}(n-2)$代表在$\tau_{n-2}$分佈中,100($1-\alpha$)百分位的值
\item 根據$SE[\hat{\beta}_{1}]$計算$t$值
\item 檢查$t$是否在$t$分佈所對應的機率是否非常小。
\end{enumerate}
\end{frame}
\subsection{實例}
\begin{frame}{\H 實例}
\begin{enumerate}
\item 建立虛無假設,例如$\hat{\beta}_{1}=0  $。以及對立假設
\item 確立$\alpha$,也就是我們容忍類型一錯誤的發生機率
\item 要很強的證據來拒斥虛無假設,例如$\alpha$=0.05.
\end{enumerate}
\end{frame}
\begin{frame}{實例}
應用\tt{R}的資料檢驗假設
\begin{description}
\item [1]例如,我們想估計$\mathtt {kid\_score}$以及$\mathtt {mom\_iq}$這兩個變數之間的線性關係,並且進行檢定。
\item [2]因為n=434, $\tau_{n-2}=\tau_{432}$。虛無分佈為$T\sim \tau_{432}$
\item [3] 查表得知當$\alpha$=0.05,臨界值{\tt qt(0.975,432)=1.96}。因此我們拒斥虛無假設,如果$|t-value|>1.96$
\item [4] $\hat{\beta}_{1}=0.6$,$SE(\hat{\beta}_{1})=0.05$,$T=\frac{0.6-0}{0.058}=10.34$,所以在$\alpha=.05$,拒斥$\emph{H}_{0}$
\end{description}
\end{frame}
\begin{frame}[fragile=singleslide]
\begin{Verbatim}[frame=single,label=R code,
fontseries=b,xleftmargin=2mm,commandchars=\\\{\},
formatcom=\color{blue}]

> m1<-with(iq,lm(kid_score ~ mom_iq))
> summary(m1)
Coefficients:
            Estimate Std. Error t value Pr(>|t|)    
(Intercept) 25.79978    5.91741 \textcolor{red}{4.36}    1.63e-05 ***
mom_iq       0.60997    0.05852 \textcolor{red}{10.42}  < 2e-16 ***
---
Signif. codes:  0 ‘***’ 0.001 ‘**’ 0.01 ‘*’ 0.05 ‘.’ 0.1 ‘ ’ 1

Residual standard error: 18.27 on \textcolor{red}{432} degrees of freedom
Multiple R-squared:  0.201,	Adjusted R-squared:  0.1991 
F-statistic: 108.6 on 1 and 432 DF,  p-value: < 2.2e-16
\end{Verbatim}
\end{frame}
\begin{frame}{\H 拒斥區域}
\begin{center}
\begin{figure}
\includegraphics[scale=.46]{"reject1.jpg"}
\end{figure}
\end{center}
\end{frame}
\subsection{p值}
\begin{frame}[fragile=singleslide]{p-value}
\begin{itemize}
\item $p$值提供我們拒斥$\emph{H}_{0}$的證據,代表假設$\emph{H}_{0}$為真時,在重複抽樣之後,觀察到至少像$t$值那麼極端的機會。
\item $p$值代表$P(|\emph{T}_{0}|>|\emph{T}_{obs}|)$也就是虛無假設下的$t$大於觀察到的$t$
\item 當$p\leq \alpha$,我們拒斥$\emph{H}_{0}$在$\alpha$的水準。\\
計算(雙尾檢定)$p$值如下:
\medskip
\begin{Verbatim}[frame=single,label=R code,
fontseries=b,xleftmargin=2mm,commandchars=\\\{\},
formatcom=\color{blue}]
>t.obs<-0.609/0.058
> pt(t.obs,df=432 , lower.tail=F)
[1] 2.006203e-23
\end{Verbatim}
\end{itemize}
\end{frame}
\begin{frame}[fragile=singleslide]
\begin{Verbatim}[frame=single,label=R code,
fontseries=b,xleftmargin=2mm,commandchars=\\\{\},
formatcom=\color{blue}]

> m1<-with(iq,lm(kid_score ~ mom_iq))
> summary(m1)
Coefficients:
            Estimate Std. Error t value Pr(>|t|)    
(Intercept) 25.79978    5.91741 4.36    \textcolor{red}{1.63e-05} ***
mom_iq       0.60997    0.05852 10.42  \textcolor{red}{< 2e-16} ***
---
Signif. codes:  0 ‘***’ 0.001 ‘**’ 0.01 ‘*’ 0.05 ‘.’ 0.1 ‘ ’ 1

Residual standard error: 18.27 on 432 degrees of freedom
Multiple R-squared:  0.201,	Adjusted R-squared:  0.1991 
F-statistic: 108.6 on 1 and 432 DF,  p-value: < 2.2e-16
\end{Verbatim}
\end{frame}
\subsection{實例}
\begin{frame}{沸點與氣壓}
\begin{itemize}
\item 1880年代,Forbes 想研究能否用不同海拔的沸點測量大氣壓力。
\item 他收集了17筆資料沸點與大氣壓力的資料,想知道沸點是否與大氣壓力相關?
\item 兩者的關係可寫成如下:
\bigskip
\begin{center}
$\textrm{log}_{10}$大氣壓力=$\beta_{0}+\beta_{1}$沸點
\end{center}
\end{itemize}
\end{frame}
\begin{frame}{Forbes}
\begin{center}
\begin{tikzpicture}
\begin{axis}[ytick={132,134, 136, 138, 140, 142, 144, 146, 148, 150}, 
xlabel=Adjusted boiling point of water ,
ylabel=$\textrm{log}_{10}$(Atmospheric pressure),
legend cell align=left,
    legend pos=north west]
\addplot[only marks, mark=*, blue!80!black] table[row sep=\\]{
 X Y \\
194.5 131.79 \\
194.3 131.79 \\
197.9 135.02 \\
198.4 135.55 \\
199.4 136.46 \\
199.9 136.83 \\
200.9 137.82 \\
201.1 138 \\
201.4 138.06 \\
201.3 138.04 \\
203.6 140.04 \\
204.6 142.44 \\
209.5 145.47 \\
208.6 144.34 \\
210.7 146.3 \\
211.9 147.54 \\
212.2 147.8 \\					
 };
  %\addplot[only marks, mark=*, red!80!black]table {\forbes};
  \addlegendentry{data}
 \addplot[no markers, very thick, dashed, red] table [row sep=\\,
 y={create col/linear regression={y=Y}}]{
 X Y \\
194.5 131.79 \\
194.3 131.79 \\
197.9 135.02 \\
198.4 135.55 \\
199.4 136.46 \\
199.9 136.83 \\
200.9 137.82 \\
201.1 138 \\
201.4 138.06 \\
201.3 138.04 \\
203.6 140.04 \\
204.6 142.44 \\
209.5 145.47 \\
208.6 144.34 \\
210.7 146.3 \\
211.9 147.54 \\
212.2 147.8 \\
 }; 
\addlegendentry{%
    $\pgfmathprintnumber{\pgfplotstableregressionb}
    \pgfmathprintnumber[print sign]{\pgfplotstableregressiona} \cdot x$} %
\end{axis}
\end{tikzpicture}
\end{center}
\end{frame}
\begin{frame}{估計}
\begin{itemize}
\item $\hat{\beta_{1}}$為0.89
\item $\reallywidehat{SE}[\hat{\beta}_{1}|X]=0.01$
\item 計算$\beta_{1}$的$t$值
\begin{align*}
t & =\hat{\beta_{1}}-\beta_{1}^{*}/\reallywidehat{SE}[\hat{\beta}_{1}|X] \\
  & = (0.9-0)/0.01 \\
  & \approx 54.4
\end{align*}
\item  我們可以拒絕接受沸點的係數等於0的假設。
\end{itemize}
\end{frame}
\subsection{單尾或雙尾檢定}
\begin{frame}[fragile=singleslide]{\H 單尾雙尾檢定}
\begin{itemize}
\item $\emph{H}_{0}=0$以及$\emph{H}_{1}\neq 0$稱為雙尾檢定,因為虛無假設可能從兩個方向檢定($\beta_{1}>0$或是$\beta_{1}<0$)
\item $\emph{H}_{0}=0$以及$\emph{H}_{1}> 0$稱為單尾檢定,因為虛無假設只能從一個方向檢定
\item 雙尾檢定的拒斥區比較小,因為分佈在兩邊。$\alpha=0.05$時需要除以2(0.025)
\begin{Verbatim}[frame=single,label=R code,
fontseries=b,xleftmargin=2mm,commandchars=\\\{\},
formatcom=\color{blue}]
> -qt(0.025, 432)
[1] 1.965471
> -qt(0.05, 432)
[1] 1.648388
\end{Verbatim}
\item 因為雙尾檢定的$t$值較大,所以$T$必須更大才能拒斥虛無假設。
\end{itemize}
\end{frame}

\section{區間估計}
\begin{frame}{\H 信賴區間}
因為
\begin{center}
$T\equiv \mathlarger{\frac{\hat{\beta}_{1}-\beta_{1}}{\widehat{SE}[\hat{\beta}_{1}]}}\sim \tau_{n-2}$
\end{center}
$P$值可表示為:
\begin{center}
$P\Big(-t_{\alpha/2}(n-2)\leq \frac{\hat{\beta}_{1}-\beta_{1}}{\widehat{SE}[\hat{\beta}_{1}]}\leq t_{\alpha/2}(n-2)\Big)=1-\alpha$
\end{center}
其中$t_{\alpha/2}(n-2)$代表在$\tau_{n-2}$分佈中,100($1-\alpha$)百分位的值
\end{frame}
\begin{frame}{\H 信賴區間}
在實例中,n=432, $\alpha$=0.05,$P$值可表示為:
\begin{center}
$P\Big(-1.96\leq \frac{\hat{\beta}_{1}-\beta_{1}}{\widehat{SE}[\hat{\beta}_{1}]}\leq 1.96\Big)=0.95$
\end{center}
改寫成:
\begin{center}
$P\Big(\hat{\beta}_{1}-t_{\alpha/2}(n-2)\reallywidehat{SE}[\hat{\beta}_{1}]\leq  \beta_{1}\leq \hat{\beta}_{1}+t_{\alpha/2}(n-2)\reallywidehat{SE}[\hat{\beta}_{1}]\Big)=1-\alpha$
\end{center}
也就是
\begin{center}
$0.609\pm 1.96\cdot 0.058=[0.49,0.72]$
\end{center}
\end{frame}
\begin{frame}[fragile=singleslide]
\begin{Verbatim}[frame=single,label=R code,
fontseries=b,xleftmargin=2mm,commandchars=\\\{\},
formatcom=\color{blue}]
> m1<-with(iq,lm(kid_score ~ mom_iq))
>n<-length(mom_iq)
>betahat1=coef(m1)[1];betahat2=coef(m1)[2]
>sigmahat=sqrt(sum(resid(m1)^2)/(n-2))
>SE1=sigmahat*sqrt(sum(mom_iq^2)/(n*sum((mom_iq-mean(mom_iq))^2)))
>SE2=sigmahat/sqrt(sum((mom_iq-mean(mom_iq))^2))
>tstar=qt(1-0.05/2,df=n-2)
>c(betahat1-tstar*SE1,betahat1+tstar*SE1)
>c(betahat2-tstar*SE2,betahat2+tstar*SE2)
\textcolor{red}{(Intercept) (Intercept)} 
\textcolor{red}{    14.16928    37.43028 }
> c(betahat2-tstar*SE2,betahat2+tstar*SE2)
\textcolor{red}{   mom_iq    mom_iq }
\textcolor{red}{0.4949534 0.7249957  }
\end{Verbatim}
\end{frame}
\begin{frame}{Forbes實例}
\begin{itemize}
\item 區間估計可寫成
\[\hat{\beta}_{1}-t_{\alpha/2}(n-2)\reallywidehat{SE}[\hat{\beta}_{1}]\leq \beta_{1}\leq
\hat{\beta}_{1}+t_{\alpha/2}(n-2)\reallywidehat{SE}[\hat{\beta}_{1}] \]
\item $\hat{\beta_{1}}=0.89$
\item $\reallywidehat{SE}[\hat{\beta}_{1}|X]=0.01$
\item 自由度為17-2=15
\item 所以 對應的 $t$ 分佈的標準差為2.13
\item 所以區間估計為
\[0.89-2.13(0.01)\leq \beta_{1}\leq 0.89+2.13(0.01)\]
\begin{center}
$0.86\leq \beta_{1}\leq 0.93$
\end{center}
\end{itemize}
\end{frame}
\section{樣本規模}
\subsection{有效估計}
\begin{frame}{\H 樣本規模}
樣本數到一定規模時,係數估計是真實參數的有效估計,也就是$\hat{\beta}_{1}$最終會等於$\beta_{1}$,圖形上為一直線。
\begin{theorem}[Consistency of OLS estimator]
當$n\rightarrow \infty$,
\begin{gather}
\plim_{n\to\infty}\hat{\beta}_{1}=\beta_{1}\notag
\end{gather}
\end{theorem}
\begin{itemize}
\item 可證明plim$\hat{\beta}_{1}=\beta_{1}+\frac{Cov[X,u]}{Var[X]}$
\item 如果$Cov[X,u]\neq 0$,OLS將不是有效(且無偏)的估計\\
\item 如果$Cov[X,u]\geq 0$,OLS估計將比真實參數$\beta_{1}$來得大。
\end{itemize}
\end{frame}
\subsection{常態分佈}
\begin{frame}{\H 常態分佈}
\begin{theorem}[Asymptotic Normality of OLS Estimator]
當樣本規模大時,樣本統計會成常態分佈
\begin{center}
$\mathlarger{\frac{\hat{\beta}_{1}-\beta_{1}}{\widehat{SE}[\hat{\beta}_{1}]}}\sim N(0,1)$\\
$\widehat{SE}[\hat{\beta}_{1}]=\mathlarger{\frac{\hat{\sigma}_{u}}{\sqrt{\sum_{i=1}^n(x-\bar{x})^2}}}$\\
$\hat{u}^2=\frac{1}{n}\sum\limits_{i=1}^n\hat{u}_{i}^2\rightarrow \sigma^2_{u}$
\end{center}
\end{theorem}
\end{frame}
\begin{frame}{\H 常態分佈}
\begin{itemize}
\item 根據中央極限定理,$\hat{\beta}_{1}-\beta_{1}/\widehat{SE}[\hat{\beta}_{1}]$會成常態分佈
\item 但是只要樣本規模夠大,即使不成常態分佈,$t$值、$p$值、信賴區間等等仍然有效。
\item 只要假設1到5成立,樣本數大的時候,OLS估計有最小的變異數,不需要成常態分佈
\end{itemize}
\end{frame}
\section{迴歸推論}
\begin{frame}{\H 對於迴歸推論的認識}
\begin{itemize}
\item 透過6個假設,讓我們了解迴歸的統計性質
\item 而迴歸推論有兩種:
\begin{enumerate}
\item 描述
\begin{itemize}
\item 藉由迴歸線描述變數之間的關係
\item 但是與母體的特性無關
\item 只有假設3(自變數有變異量)必須成立
\end{itemize}
\item 預測推論
\begin{itemize}
\item 預測新的觀察值
\item 例如X是教育程度,Y是收入,可預測如果教育程度為高中程度時,收入可能為多少?
\item 假設2(隨機抽樣)以及假設3(自變數有變異量)必須成立
\item 假設1(線性)也需要成立。
\end{itemize}
\end{enumerate}
\end{itemize}
\end{frame}
\section{迴歸預測}
\begin{frame}{迴歸預測}
\begin{itemize}
\item 求出$\beta$之後,可進行預測(prediction)以及計算預測值(predicted values)。
\item 預測值則是根據信賴區間得到的值。
\item 預測的範圍稱為Prediction interval或PI,大於預測值的信賴區間或CI。
\end{itemize}
\end{frame}
\subsection{預測}
\begin{frame}{預測}
\begin{itemize}
\item 預測指的是還未發生,需要考慮$\sigma^2$,也就是真正觀察的值可能跟從代入$\beta$得到的值之間有一定
的誤差。
\[\tilde{y_{*}}=\hat{\beta_{0}}+\hat{\beta_{1}}x_{*}\]
\item $\tilde{y_{*}}$是我們根據迴歸係數得到的預測值,但是真正的值有可能等於
\[ y_{*}=\beta_{0}+\beta_{1}x_{*}+e_{*} \]
\item 計算$\tilde{y_{*}}$的標準誤公式為:
\[\sigma^2+\sigma^2\Big(\frac{1}{n}+\frac{(x_{*}-\bar{x})^2}{SXX}\Big)\]
\item 如果$x_{*}$跟$x_{i}$差別很大,那麼$\sigma^2(\frac{1}{n}+\frac{(x_{*}-\bar{x})^2}{SXX})$影響標準誤的程度將高於原本的$\sigma^2$,也就是來自於殘差的誤差。
\end{itemize}
\end{frame}
\begin{frame}{預測的標準誤以及區間估計}
\begin{itemize}
\item 估計標準誤的公式為
\[ \textrm{se.pred}(\tilde{y_{x}}|x_{*})=\sigma\sqrt{1+1/n+(x_{*}-\bar{x})^2/SXX}\]

\item $\sigma$來自於 Residual standard error
\item 預測的區間估計可寫成
\[ \hat{\beta_{0}}+\hat{\beta_{1}}x_{*}\pm t_{\alpha/2, n-2}\times \textrm{se.pred}(\tilde{y}|x)\]
\end{itemize}
\end{frame}
\subsection{預測值}
\begin{frame}{預測值的標準誤與信賴區間}
\begin{itemize}
\item 按照$\beta$代入已經存在的$x$
\item 預測值可以寫成
\[\hat{y}=\beta_{0}+\beta_{1}x_{*}\]
\item $\hat{y}$ 的標準誤寫成
\[\textrm{se.fit}(\hat{y}|x_{*})=\hat{\sigma}(\frac{1}{n}+\frac{x_{i}-\bar{x}}{SXX})^{1/2}\]
\item 預測值的信賴區間表示為:
\[[\hat{y}-t_{(\alpha/2, n-2)}\sigma\cdot \textrm{se.fit},\hspace{1em} 
\hat{y}+t_{(\alpha/2, n-2)}\sigma\cdot \textrm{se.fit}] \]
\end{itemize}
\end{frame}
\subsection{實例}
\begin{frame}{PI, CI}
\begin{itemize}
\item 假設有一筆資料,n=11,$\sum X_{i}=292.9$,$\sum Y_{i}=69.03$,$\sum X_{i}^2=8141.75$,
$\sum X_{i}Y_{i}=1890.2$,$\sum Y_{i}^2=442.19$,$E[Y|X]=2.224+0.152X$,$S=0.344$,$\bar{X}=26.627$。
\item 點估計:$E[Y|X=25]=2.224+0.152*25=6.08$
\item 預測區間:$t_{(0.025, 9)}(0.344)\sqrt{1+\frac{1}{11}+\frac{11(25-26.627)^2}{11\cdot 8141.75-292.9^2}}$
\item 預測區間值PI=6.028$\pm$0.8165
\item 預測值:X=25
\item 信賴區間:$t_{(0.025, 9)}0.3444\sqrt{\frac{1}{11}+\frac{11(25-26.627)^2}{11(8141.75)-(292.9)^2}}$=
$0.262\times 0.1082=0.244$
\item CI=6.028$\pm$ 0.244
\end{itemize}
\end{frame}
\begin{frame}{\H 迴歸推論}
\begin{enumerate}
\setcounter{enumi}{2}
\item 因果推論
\begin{itemize}
\item 推論並不存在的事實(counterfactual)
\item 例如:假如我的教育程度「只有」小學程度,我的收入為何?
\item 假設1到4(X條件下的誤差項為0)必須成立 
\item 假設5(誤差項的變異數同質性)不一定需要成立,仍然可以進行因果推論
\end{itemize}
\end{enumerate}
\end{frame}
\begin{frame}{\H 如何解釋迴歸係數}
\begin{itemize}
\item 盡量用X與Y相關、或是X增加預期Y會增加或減少。
\item 不要用「影響」、「導致」等動詞
\item 瞭解每一單位的X及Y代表的意義
\item 簡短地提到統計顯著水準
\item 討論係數的大小的意義
\end{itemize}
\end{frame}
\begin{frame}
\begin{center}
\begin{figure}
\includegraphics[scale=.4]{substantive1.jpg}
\end{figure}
\end{center}
\end{frame}

\begin{frame}
\begin{center}
\begin{figure}
\includegraphics[scale=.4]{substantive2.jpg}
\end{figure}
\end{center}
\end{frame}
\section{結論}
\begin{frame}\frametitle{\H 總結}
\begin{enumerate}
\item {\K 瞭解如何進行假設檢定}
\item {\K 瞭解如何估計信賴區間}
\item {\K 瞭解各種迴歸假設的用途}
\item {\K 瞭解迴歸係數的詮釋}
\end{enumerate}
\end{frame}

\end{document}
